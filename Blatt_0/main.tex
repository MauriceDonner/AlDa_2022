
\documentclass{article}

% Deactivate sectsty warning when loading sectsty {{{
\usepackage[immediate]{silence}
\WarningFilter[temp]{latex}{Command}
\usepackage{sectsty}
    \sectionfont{\normalfont\sffamily\bfseries\color{blue!40!black}}
    \subsectionfont{\normalfont\sffamily\bfseries\color{blue!30!black}}
\DeactivateWarningFilters[temp]
\makeatletter % disable the runtime redefinitions
\let\SS@makeulinesect\relax
\let\SS@makeulinepartchap\relax
\makeatother
% }}}

\usepackage[margin=4cm]{geometry}
    \setlength\parindent{0pt}
\usepackage{fancyhdr}
    \pagestyle{fancy}
\usepackage{fontspec}
    \setsansfont{Linux Biolinum O}
\usepackage{polyglossia}
    \setmainlanguage{german}
\usepackage{sectsty}
    \sectionfont{\normalfont\sffamily\bfseries\color{blue!40!black}}
    \subsectionfont{\normalfont\sffamily\bfseries\color{blue!30!black}}
\usepackage{amsmath}
\usepackage{amssymb}
\usepackage{siunitx}
\usepackage{float}
\usepackage{booktabs}
\usepackage{subcaption}
\usepackage{graphicx}
\usepackage{xcolor}
\usepackage{listings}
    \lstset{language=Python,
	basicstyle=\footnotesize\ttfamily,
	breaklines=true,
	framextopmargin=50pt,
	frame=bottomline,
	backgroundcolor=\color{white!86!black},
	commentstyle=\color{blue},
	keywordstyle=\color{red},
	stringstyle=\color{orange!80!black}}
\usepackage{tikz}

\title{\textsf{\color{blue!40!black}Übungsblatt 0}}
\author{Maurice Donner \and Jan Hubrich \and Adrian Müller}

\begin{document}

\maketitle

\section{Beweis - $A \setminus B = A \cap \bar B$}
\begin{flalign*}
    \text{Def 1.}\quad & A \setminus B = \left\{ x \in U \mid x \in A \land x \not\in B \right\}&&\\
    \text{Def 2.}\quad & \bar B = \left\{ x \in U \mid x \not\in B \right\} &&\\
    \text{Def 3.}\quad & A \cap B = \left\{ x \in U \mid x \in A \land x \in B \right\} &&\\[.5cm]
    \text{1.}\quad & \text{Sei}\ x \in A \setminus B \ \text{beliebig} &&\\
    \text{2.}\quad & x \not\in B & (1) (\text{Def. 1})&&\\
    \text{3.}\quad & x \in \bar B & (2) ( \text{Def. 2} )&&\\
    \text{4.}\quad & x \in A \land x \in \bar B & (1) (3)&&\\
    \text{5.}\quad & \left( A \setminus B \right) = A \cap \bar B & (1) (4) ( \text{Def. 3})&&\\
\end{flalign*}

\section{Beweis $(A \setminus B) \cup B = A \cup B$} 

\begin{align*}
    \text{Def 1.}\quad & \left( A \setminus B \right) \cup B =
    \left\{ x \mid x \in \left( A \land x \not\in B \right) \lor
	\left( x \in B \right) \right\} &&\\
    \text{Def 2.} \quad & A \cup B = \left\{ x \mid x \in A \lor x \in B \right\}&&\\[0.5cm]
    \text{1.}\quad & \text{Sei} \ x \in (A \setminus B) \cup B &&\\
    \text{2.}\quad & (\text{1. Fall}) \ ( x \in A ) \land ( x \not\in B ) &
    (\text{Def.} \ 1, \text{case} \ 1) &&\\
    \text{3.}\quad & x \in A \cup B & (\text{Def.} 2) (2)&&\\
    \text{4.}\quad & ( \text{2. Fall}) \ x \in B &
    (\text{Def.} \ 1, \text{case} \ 2) &&\\
    \text{5.}\quad & x \in A \cup B & (\text{Def.} \ 2) (4) &&\\
    \text{6.} \quad & \forall x \in \left( \left( A \setminus B \right) \cup B \right) :
    x \in \left( A \cup B \right) & (1 \text{-} 5)&&\\
    \text{7.} \quad & \left( A \setminus B \right) \cup B = A \cup B & (1 \text{-} 5) (\text{us} \forall)
\end{align*}

\section{Beweis durch vollständige Induktion} 
Zeige, dass
\begin{align}
    \label{3}
    \sum_{i=0}^n 2 ^{i} = 2 ^{n+1} -1 \quad \text{für alle}\ x \in \mathbb{N}_0
\end{align}
Grenzfall \( n=0 \)
\begin{align}
    2 ^{0} = 1 = 2 ^{0+1} - 1
\end{align}
ist wahr.
angenommen, (\ref{3}) ist wahr. beweise, dass (\ref{3}) auch für \( n+1 \) gilt:
\begin{align}
    \sum_{i=0} ^{n+1} 2 ^{i} &= \sum_{i = 0} ^{n} 2 ^{i} + 2 ^{n+1} \label{ass1}\\
    &= 2 ^{n+1} -1 + 2 ^{n+1} & (\text{ass. 1} )&&\\
    &= 2 \cdot \left( 2 ^{n+1} \right) -1 = 2 ^{n+2} - 1
\end{align}

\section{Beweis durch vollständige Induktion}
Zeige, dass für \( \left\{ n \in \mathbb{N} \mid n \geq 4 \right\} \) gilt:
\begin{align}
\label{ass2}
    2 ^{n} < n! 
\end{align}
Grenzfall \( n = 4 \)
\begin{align}
    2 ^{4} = 16 < 4! = 24
\end{align}
angenommen, (\ref{ass2}) ist wahr. beweise, dass (\ref{ass2}) auch für \( n+1 \) gilt:
\begin{align}
    2 ^{n+1} = 2 \cdot 2 ^{n} < n! \cdot 2 < n! \cdot \left( n+1 \right) =
    \left( n+1 \right) ! \quad \forall n \geq 4 
    & & (\text{ass.}\ \ref{ass2})
\end{align}
\end{document}

